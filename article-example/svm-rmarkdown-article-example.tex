\documentclass[11pt,]{article}
\usepackage[left=1in,top=1in,right=1in,bottom=1in]{geometry}
\newcommand*{\authorfont}{\fontfamily{phv}\selectfont}
\usepackage[]{libertine}


  \usepackage[T1]{fontenc}
  \usepackage[utf8]{inputenc}




\usepackage{abstract}
\renewcommand{\abstractname}{}    % clear the title
\renewcommand{\absnamepos}{empty} % originally center

\renewenvironment{abstract}
 {{%
    \setlength{\leftmargin}{0mm}
    \setlength{\rightmargin}{\leftmargin}%
  }%
  \relax}
 {\endlist}

\makeatletter
\def\@maketitle{%
  \newpage
%  \null
%  \vskip 2em%
%  \begin{center}%
  \let \footnote \thanks
    {\fontsize{18}{20}\selectfont\raggedright  \setlength{\parindent}{0pt} \@title \par}%
}
%\fi
\makeatother




\setcounter{secnumdepth}{0}




\title{Detection of Pluggable Transport Techniques
Report \thanks{Document format based on svmille template
(\url{http://github.com/svmiller}).}  }
 



\author{\Large Zuhayr Ansari (bfq6fs), Luke Del Giudice
(wcc5ub)\vspace{0.05in} \newline\normalsize\emph{University of
Virginia}  }


\date{}

\usepackage{titlesec}

\titleformat*{\section}{\normalsize\bfseries}
\titleformat*{\subsection}{\normalsize\itshape}
\titleformat*{\subsubsection}{\normalsize\itshape}
\titleformat*{\paragraph}{\normalsize\itshape}
\titleformat*{\subparagraph}{\normalsize\itshape}


\usepackage{natbib}
\bibliographystyle{apsr}
\usepackage[strings]{underscore} % protect underscores in most circumstances



\newtheorem{hypothesis}{Hypothesis}
\usepackage{setspace}


% set default figure placement to htbp
\makeatletter
\def\fps@figure{htbp}
\makeatother

\usepackage{hyperref}
\usepackage{array}
\usepackage{caption}
\usepackage{graphicx}
\usepackage{siunitx}
\usepackage[table]{xcolor}
\usepackage{multirow}
\usepackage{hhline}
\usepackage{calc}
\usepackage{tabularx}
\usepackage{fontawesome}
\usepackage[para,online,flushleft]{threeparttable}

% move the hyperref stuff down here, after header-includes, to allow for - \usepackage{hyperref}

\makeatletter
\@ifpackageloaded{hyperref}{}{%
\ifxetex
  \PassOptionsToPackage{hyphens}{url}\usepackage[setpagesize=false, % page size defined by xetex
              unicode=false, % unicode breaks when used with xetex
              xetex]{hyperref}
\else
  \PassOptionsToPackage{hyphens}{url}\usepackage[draft,unicode=true]{hyperref}
\fi
}

\@ifpackageloaded{color}{
    \PassOptionsToPackage{usenames,dvipsnames}{color}
}{%
    \usepackage[usenames,dvipsnames]{color}
}
\makeatother
\hypersetup{breaklinks=true,
            bookmarks=true,
            pdfauthor={Zuhayr Ansari (bfq6fs), Luke Del Giudice
(wcc5ub) (University of Virginia)},
             pdfkeywords = {},  
            pdftitle={Detection of Pluggable Transport Techniques
Report},
            colorlinks=true,
            citecolor=blue,
            urlcolor=blue,
            linkcolor=magenta,
            pdfborder={0 0 0}}
\urlstyle{same}  % don't use monospace font for urls

% Add an option for endnotes. -----


% add tightlist ----------
\providecommand{\tightlist}{%
\setlength{\itemsep}{0pt}\setlength{\parskip}{0pt}}

% add some other packages ----------

% \usepackage{multicol}
% This should regulate where figures float
% See: https://tex.stackexchange.com/questions/2275/keeping-tables-figures-close-to-where-they-are-mentioned
\usepackage[section]{placeins}


\begin{document}
	
% \pagenumbering{arabic}% resets `page` counter to 1 
%    

% \maketitle

{% \usefont{T1}{pnc}{m}{n}
\setlength{\parindent}{0pt}
\thispagestyle{plain}
{\fontsize{18}{20}\selectfont\raggedright 
\maketitle  % title \par  

}

{
   \vskip 13.5pt\relax \normalsize\fontsize{11}{12} 
\textbf{\authorfont Zuhayr Ansari (bfq6fs), Luke Del Giudice
(wcc5ub)} \hskip 15pt \emph{\small University of Virginia}   

}

}








\begin{abstract}

    \hbox{\vrule height .2pt width 39.14pc}

    \vskip 8.5pt % \small 

\noindent Censorship suppresses the free distribution of information to
protect users from harmful material. What material is deemed harmful by
the governments and private organizations that conduct censorship varies
widely and often is motivated by a need to exert political and social
control. The right to access information anonymously is therefore at
odds with limiting the spread of information deemed detrimental to
society. One censorship circumvention technique is pluggable transport,
using domain fronting to connect to allowed websites and then being
redirected to a desired destination. Snowflake is an implementation of
this diverting pluggable transport strategy, using domain fronting to a
broker and receiving a snowflake proxy to connect to through WebRTC.
Unfortunately, recent research suggests that the DTLS handshake used by
Snowflake may be easily distinguishable from other WebRTC applications,
providing an opportunity for censors. In this project we attempted to
replicate this detection of Snowflake and alter the DTLS handshake
procedure to increase the difficulty of detection.


    \hbox{\vrule height .2pt width 39.14pc}


\end{abstract}


\vskip -8.5pt


 % removetitleabstract

\noindent  

\section{Introduction}\label{introduction}

We first encountered the concept of fingerprinting Snowflake on a
\href{https://ntc.party/t/ooni-reports-of-tor-blocking-in-certain-isps-since-2021-12-01/1477}{tor
message board} where a user referenced that Snowflake blocking is being
done in Russia by fingerprinting and blocking DTLS Handshakes. This
threat to user anonymity was obviously concerning, and further
investigation showed that academia had also taken notice of this
vulnerability in Snowflake. In particular, the work of

To test this result After investigating further, we found that these

Previous successful attempts of detecting Snowflake, including those
made by Midtlien and Palma {[}2{]} and by MacMillan et al {[}3{]}, have
guided our approaches to this project. Government censors have attempted
to block

\href{http://ajps.org/2015/03/26/the-ajps-replication-policy-innovations-and-revisions/}{``show
your work'' initiative} \emph{American Journal of Political Science}
(\emph{AJPS})

\texttt{geometry:}

Snowflake has recently been used to circumvent censorship in Russia and
Iran, whose governments have increased efforts to block access to Tor
entry relays.

\section{Results}\label{results}

\subsection{Mini Results}\label{mini-results}

The project consisted of four stages: procuring data both through self
collection and using the Macmillan database {[}3{]}, using a sklearn
random forest classifier model to detect Snowflake WebRTC handshakes,
altering the Snowflake handshakes' padding with scapy, and finally using
go language to alter the handshake packets. The MacMillan database
consisted of data of about 2000 WebRTC handshakes each from Google Meet,
Facebook, Discord, and Snowflake. Simulation of WebRTC handshakes was
done through selenium and collected with pyshark. This combined data, of
around \_\_\_\_\_ WebRTC handshakes per application, were now in a
pcapng format. These packet files were then converted to csv files, with
columns representing features, for ease of use with pandas. The source
of the packet was represented as a 1 for snowflake and a 0 for any other
application. This matrix of packets and packet features could then be
fed into an sklearn random forest classifier, where packet features were
predictors and the source of the packet was the outcome trying to be
predicted. The random forest classifier was chosen due to its robustness
to noise, improved accuracy, and reduced overfitting from its
combination of multiple decision trees. With k fold cross validation
this yielded a balanced accuracy of 99.85\%, precision of 0.9976, and a
receiver operating characteristic area under the curve, roc-auc, score
of 0.9999. Therefore our classifier was successful in distinguishing
between the WebRTC handshake of Snowflake and other applications, a
pattern that could be used by censors to detect and block Snowflake
traffic.

Now that we could detect Snowflake usage, the question was whether we
could alter this handshake to avoid detection. Giving Snowflake a new
pattern would not prevent censors from adjusting to the change and
recognizing the traffic as belonging to Snowflake. Therefore we aimed to
try and conform the Snowflake traffic to that of another user specified
application. Our previous random forest model could then be used as a
test of whether or not Snowflake was distinguishable from that
application.

For my template, I'm pretty sure this is mandatory.\footnote{The main
  reason I still use \texttt{pdflatex} (and most readers probably do as
  well) is because of LaTeX fonts.
  \href{http://www-rohan.sdsu.edu/~aty/bibliog/latex/gripe.html}{Unlike
  others}, I find standard LaTeX fonts to be appealing.}

\section{Diagram of Program}\label{diagram-of-program}

Perhaps the greatest intrigue of R Markdown comes with the
\href{http://yihui.name/knitr/}{\texttt{knitr} package} provided by
\citet{xie2013ddrk}.





\newpage
\singlespacing 
\bibliography{master.bib}

\end{document}
